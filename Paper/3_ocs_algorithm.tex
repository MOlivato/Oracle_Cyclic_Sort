
\section{FixCycle version of Cycle Sort}

% TODO: Pseudocodice del CycleSort che utilizza la funzione FixCycle

\begin{algorithm}[H]
\caption{\textsc{CycleSort}: Array sorting algorithm producing the minimum number of array writes}
\label{alg:ocs_algorithm}      
\begin{algorithmic}[1]
\Procedure{\textsc{CycleSort}}{}
	\For	{$i \gets 1$ {\bf to} $n$}
		\State $\textsc{FixCycle}(i)$
	\EndFor
\EndProcedure
\end{algorithmic}
\end{algorithm}


\section{The Oracle Cyclic Sort algorithm}

Our goal is to minimize the number of redundant comparison when there are few elements which are unfit.
Specifically, since the \textsc{FixCycle} procedure performs $n-1$ comparison
when it is given a position which is already fit,
we want to avoid calling \textsc{FixCycle} at all on fit elements.
To this end, we devise a technique to find an unfit element in the array.

\begin{thm}
\label{th_fat}
If $v[0] \le v[1] \le ... \le v[i]$ and $v[i] > v[i+1]$ then $v[i]$ is not fit.
\begin{proof}
Suppose by contradiction $\overline{v}^{-1}[v[i]] = i$.
Then, there exist exactly $i$ elements in $v$ strictly smaller than $v[i]$.
However, we have $i+1$ such elements, i.e., $v[0],\dots,v[i-1],v[i+1]$.
\end{proof}
\end{thm}

This Theorem always
applies for some $i$, unless v is already sorted. Indeed,
it is sufficient to scan the array from left
to right, and return the first index i such
that v [ i ] > v [ i + 1 ] . This sweep is conducted by

Observe that Theorem~\ref{th_fat} always applies for some $i$, unless $v$ is already sorted.
Indeed, it is sufficient to scan the array from left to right,
and return the first index $i$ such that $v[i] > v[i+1]$.
This sweep is conducted by the function \textsc{UnfitOracle},
which either returns the index of an unfit element,
or determines that the array is already sorted,
in which case it returns~$-1$.
The pseudo-code of this function is given in Algorithm~\ref{alg:unfit_oracle}.

\begin{algorithm}
\caption{\textsc{Unfit Oracle}: Oracle able to find the index of an unfit array element}
\label{alg:unfit_oracle}      
\begin{algorithmic}[1]
\Function{\textsc{UnfitOracle}}{}
	\For {$i \leftarrow 0$ {\bf to} $n - 1$}
			\If{$v[i] > v[i+1]$}
				\Return {$ i $}
	\EndIf
	\EndFor
	\Return {$ -1 $}
\EndFunction
\end{algorithmic}
\end{algorithm}

Function \textsc{UnfitOracle}, combined with the \textsc{FixCycle} procedure, yields our sorting algorithm.
This algorithm runs \textsc{UnfitOracle} repeatedly, and,
as long as there is an unfit element in the array,
it fixes the position of that element along with its permutation cycle
by calling \textsc{FixCycle}.
This algorithm is realized by the procedure \textsc{OracleCyclicSort},
whose pseudo-code is given in Algorithm~\ref{alg:ocs_algorithm}.

\begin{algorithm}
\caption*{\textbf{Oracle Ciclic Sort}: Array sorting algorithm producing the minimum number of array writes}
%\label{alg1}      
\begin{algorithmic}[1]
\Procedure{OCS}{}
	\State {\bf static} $lp \leftarrow$ 0
	\State $next \textunderscore unfit \leftarrow UO(-1,lp)$
    \While {$next \textunderscore unfit \neq -1 $}
		\State $CPS(next \textunderscore unfit, lp)$
		\State $ fixed \leftarrow next \textunderscore unfit $
		\State $ next \textunderscore unfit \leftarrow UO(fixed,lp)$
	\EndWhile
\EndProcedure
\end{algorithmic}
\end{algorithm}

