% Paper Matteo Olivato 2017
%\documentclass[twoside,twocolumn]{article}
\documentclass[twoside,twocolumn]{article}

\usepackage[sc]{mathpazo} % Use the Palatino font
\usepackage[T1]{fontenc} % Use 8-bit encoding that has 256 glyphs
\linespread{1.05} % Line spacing - Palatino needs more space between lines
\usepackage{microtype} % Slightly tweak font spacing for aesthetics

\usepackage[english]{babel} % Language hyphenation and typographical rules

\usepackage[hmarginratio=1:1,top=32mm,columnsep=20pt]{geometry} % Document margins
\usepackage[hang, small,labelfont=bf,up,textfont=it,up]{caption} % Custom captions under/above floats in tables or figures
\usepackage{booktabs} % Horizontal rules in tables

\usepackage{lettrine} % The lettrine is the first enlarged letter at the beginning of the text

\usepackage{enumerate} % Customized lists
%\setlist[itemize]{noitemsep} % Make itemize lists more compact

\usepackage{abstract} % Allows abstract customization

\renewcommand{\abstractnamefont}{\normalfont\bfseries} % Set the "Abstract" text to bold
\renewcommand{\abstracttextfont}{\normalfont\small\itshape} % Set the abstract itself to small italic text

\usepackage{titlesec} % Allows customization of titles
\renewcommand\thesection{\Roman{section}} % Roman numerals for the sections
\renewcommand\thesubsection{\roman{subsection}} % roman numerals for subsections
\titleformat{\section}[block]{\large\scshape\centering}{\thesection.}{1em}{} % Change the look of the section titles
\titleformat{\subsection}[block]{\large}{\thesubsection.}{1em}{} % Change the look of the section titles

\usepackage{fancyhdr} % Headers and footers
\pagestyle{fancy} % All pages have headers and footers
\fancyhead{} % Blank out the default header
\fancyfoot{} % Blank out the default footer
\fancyhead[C]{Oracle Cyclic Sort $\bullet$ \today} % Custom header text
\fancyfoot[RO,LE]{\thepage} % Custom footer text

\usepackage{titling} % Customizing the title section
\usepackage{caption} % Customizing the caption section

\usepackage{amsthm} % pacchetto per la scrittura di teoremi
\theoremstyle{plain}
\newtheorem{thm}{Theorem}[section] % reset theorem numbering for each subsection
\newtheorem{lemma}{Lemma}[section] % reset lemma numbering for each subsection
\newtheorem{dfn}{Definition}[section] % reset definition numbering for each subsection
\newtheorem{prop}{Proposition}

\usepackage{float}
\usepackage{algorithm}
\usepackage{algorithmicx}
\usepackage{algpseudocode}
\algdef{SE}[DOWHILE]{Do}{doWhile}{\algorithmicdo}[1]{\algorithmicwhile\ #1}%

%\usepackage{algorithmic}
%\floatname{algorithm}{Procedure}
%\renewcommand{\algorithmicrequire}{\textbf{Input:}}
%\renewcommand{\algorithmicensure}{\textbf{Output:}}

\usepackage[style=numeric-comp,backend=bibtex]{biblatex}
\usepackage{bookmark} % pacchetto usato per i segnalibri
\usepackage{graphicx} % pacchetto per inserire le immagini
\usepackage[dvipsnames]{xcolor} % pacchetto per la modifica degli stili dei colori
\usepackage{hyperref} % For hyperlinks in the PDF
\hypersetup{
  colorlinks,
  allcolors=PineGreen,
  pdftitle={Oracle Cyclic Sort: improved sorting with minimum number of writes},
  pdfauthor={Matteo Olivato, Romeo Rizzi, Massimo Cairo},
  pdfkeywords={PDF, Sort, Comparison Sorting, Cycle Sort, Minimum write sort method, Oracle Ciclic Sort},
  linktoc=page
}

% creo un nuovo comando che mi inserisce il riferimento alla bibliografia
\newcommand\mycite[1]{{\enspace\small\cite{#1}}}
% creo un nuovo comando return per il pacchetto ddegli algoritmi
\algrenewcommand\Return{\State \algorithmicreturn{} }%
%----------------------------------------------------------------------------------------
%	TITLE SECTION
%----------------------------------------------------------------------------------------

\setlength{\droptitle}{-4\baselineskip} % Move the title up

\pretitle{\begin{center}\Huge\bfseries} % Article title formatting
\posttitle{\end{center}} % Article title closing formatting
\title{Oracle Cyclic Sort: improved sorting with minimum number of writes} % Article title
\author{%
\textsc{Matteo Olivato}\thanks{A thank you or further information} \\[1ex] % Your name
\normalsize University of Verona \\ % Your institution
\normalsize \href{mailto:matteo.olivato@studenti.univr.it}{matteo.olivato@studenti.univr.it} % Your email address
\and % Uncomment if 2 authors are required, duplicate these 4 lines if more
\textsc{Romeo Rizzi}\thanks{Corresponding author} \\[1ex] % Second author's name
\normalsize University of Verona \\ % Second author's institution
\normalsize \href{mailto:romeo.rizzi@univr.it}{romeo.rizzi@univr.it} % Second author's email address
\and % Uncomment if 2 authors are required, duplicate these 4 lines if more
\textsc{Massimo Cairo}\thanks{Corresponding author} \\[1ex] % Second author's name
\normalsize University of Verona \\ % Second author's institution
\normalsize \href{mailto:massimo.cairo@univr.it}{massimo.cairo@univr.it} % Second author's email address
%\and % Uncomment if 2 authors are required, duplicate these 4 lines if more
%\textsc{Jane Smith}\thanks{Corresponding author} \\[1ex] % Second author's name
%\normalsize University of Utah \\ % Second author's institution
%\normalsize \href{mailto:jane@smith.com}{jane@smith.com} % Second author's email address
}
\date{\today} % Leave empty to omit a date
\renewcommand{\maketitlehookd}{%
\begin{abstract}
% Abstract

\noindent The {\sc CyclicSort} algorithm takes $\Theta(n^2)$ comparisons in order 
to sort an array $v$ of $n$ integers.

Even in the best case, when given an already sorted array $v$, {\sc 
CyclicSort} still employs at least $n(n-1)$ comparisons.

Yet, until now, {\sc CyclicSort} was the only sorting algorithm that, 
assuming $O(1)$ working memory,

could guarantee the minimum possible number of writes on the array $v$.

And, since this property may turn out to be the most significant factor 
in certain niche applications, {\sc Cyclic Sort} retains its own (though 
modest) place in the Olympus of sorting algorithms.

We propose smarter and more performant algorithms with this property, 
and a first rough analysis of the issue of minimizing the number of 
write operations within the array $v$ to be sorted.

One basic tool in our approach is an oracle subroutine which returns a 
position $i$ such that element $v[i]$ is not at its correct place, or a 
check that array $v$ is already sorted. This subroutine offers a nice 
decomposition opportunity for this problem.
\end{abstract}
}

% Aggiungo la bibliografia
\addbibresource{bibliography/bibliography.bib}

\begin{document}
% Print the title
\maketitle

%----------------------------------------------------------------------------------------
%	ARTICLE CONTENTS
%----------------------------------------------------------------------------------------
% Paper introduction to minimum write sorting algorithm

\section{Introduction}

The problem of sorting with the minimum number of writes was studied for the first time by the Cycle Sort\mycite{haddon1990cycle} algorithm. The core of the algorithm is based on the concept of in-place cyclic permutation\mycite{duijvestijn1972correctness} which allows to obtain the theoretical minimum number of writes.

The Cycle Sort provides a solution for this problem being able to identify every different permutation cycle to run to sort the array. The OCS algorithm is suitable for environments with little memory, as it uses costant space, and environments where memory writes have high cost, in terms of time, energy or memory life.
The main problem of the algorithm in terms of complexity is that it can't recognize a global ordered element when it arrives on it, so it must compare the element with all its successors in the last part of the array. This involves a high number of unnecessary comparisons which increase the complexity of the algorithm.

Our goal is to pass quickly on elements already sorted and find an elements of a permutation cycle until the array is not sorted. 

For simplicity our case study takes into consideration the sorting problem for arrays of distinct integers.

% The OCS Algorithm
\section{Preliminaries and notation}

The task is to sort an array $v$ of $n$ elements taken from a totally ordered universe.
To ease our exposition, we assume that all elements in $v$ differ, so that the intended output vector $\overline{v}$ is unique. 
Also, the inverse maps $v^{-1}$ and $\overline{v}^{-1}$ are well-defined.
For $i=1,\dots,n$, position $i$ and element $v[i]$ are called \emph{fit} if $v[i]=\overline{v}[i]$, that is, if element $v[i]$ is already in its final position. 
The specificity of the Cycle Sort algorithm is that is uses the minimum possible number of writes: no fit element gets ever moved and, when an unfit element $v[i]$ is moved, it must be placed directly in its final position $\overline{v}^{-1}[v[i]]$ and becomes fit. If we consider the graph having the set of positions $\{1,\ldots,n\}$ as its nodes, and with arc set $\{(i,\overline{v}^{-1}[v[i]]) \, : \, i=1,\ldots, n\}$, then it comprises a family of directed cycles, with each node belonging to exactly one cycle of the family (the fit positions are those holding self-loops); this is the well know representation of a permutation as a family of node-disjoint covering cycles. Given our commitment to minimize the number of writes, i.e., to achieve precisely one write/movement for every unfit element, then, when we are about to move an unfit element $v[i]$, we are due to settle all the elements of its cycle.

We delegate this task to the following procedure, called \textsc{FixCycle}, which encapsulates the heart of the previous Cycle Sort algorithm (but will be managed more subtly and efficiently in our new proposal).

The \textsc{FixCycle} procedure is given the index of an element to place in its final position.
To find its final position, say i, it counts all the smaller
elements present in the array (spending $n-1$ comparisons).
Then, it places the element in position $i$, and it recurs to place the
element previously in position $i$.

The procedure stops when it finds an element which must be moved into the initial position, closing the cycle.
The pseudo-code of this procedure is provided for reference in Algorithm~\ref{alg:fix_cycle}.

Only procedure \textsc{FixCycle} is allowed to write elements in the array, so the final number of writes is precisely the number of writes made by this procedure, which is known to be optimal for the purpose of sorting an array.
Our purpose is to improve on the number of comparisons.

\begin{algorithm}[H]
\caption{\textsc{CorrectPositionOf}: Procedure that find the correct position of an element of v}
\label{alg:corposof}      
\begin{algorithmic}[1]
\Procedure{\textsc{CorrectPositionOf}}{$i$, $val$}
	% Count the number $k$ of elements $ i < val$
	\State $k \gets 1$
	
	\For	{$j \gets 1$ {\bf to} $n$}
		\If{$j \neq i$ {\bf and} $v[j] < val$}
			\State $k \gets k + 1$
		\EndIf
	\EndFor
	\Return {$ k $}
\EndProcedure
\end{algorithmic}
\end{algorithm}

\begin{algorithm}[H]
\caption{\textsc{FixCycle}: Ciclic permutation sorting procedure}
\label{alg:fix_cycle}      
\begin{algorithmic}[1]
\Procedure{\textsc{FixCycle}}{$i$}
    \State $val \gets v[i]$
	\Do
		
		\State $cpos \gets \textsc{CorrectPositionOf}(i, val)$
		% Place $val$ in the final position $k$, and pick the next element.
		\State $temp \gets v[cpos]$
		\State $v[cpos] \gets val$
		\State $val \gets temp$
	\doWhile{$cpos \neq i$}
\EndProcedure
\end{algorithmic}
\end{algorithm}

\begin{lemma}
\label{lm_fixcycle}
A sorting algorithm performs the minimum number of writes within 
vector $v$ if all its writes are performed within calls to the \textsc{FixCycle}
procedure.
\end{lemma}
\section{FixCycle version of Cycle Sort}

% TODO: Pseudocodice del CycleSort che utilizza la funzione FixCycle

\begin{algorithm}[H]
\caption{\textsc{CycleSort}: Array sorting algorithm producing the minimum number of array writes}
\label{alg:ocs_algorithm}      
\begin{algorithmic}[1]
\Procedure{\textsc{CycleSort}}{}
	\For	{$i \gets 1$ {\bf to} $n$}
		\State $\textsc{FixCycle}(i)$
	\EndFor
\EndProcedure
\end{algorithmic}
\end{algorithm}



\section{The Oracle Cyclic Sort algorithm}

Our goal is to minimize the number of redundant comparison when there are few elements which are unfit.
Specifically, since the \textsc{FixCycle} procedure performs $n-1$ comparison
when it is given a position which is already fit,
we want to avoid calling \textsc{FixCycle} at all on fit elements.
To this end, we devise a technique to find an unfit element in the array.

\begin{thm}
\label{th_fat}
If $v[0] \le v[1] \le ... \le v[i]$ and $v[i] > v[i+1]$ then $v[i]$ is not fit.
\begin{proof}
Suppose by contradiction $\overline{v}^{-1}[v[i]] = i$.
Then, there exist exactly $i$ elements in $v$ strictly smaller than $v[i]$.
However, we have $i+1$ such elements, i.e., $v[0],\dots,v[i-1],v[i+1]$.
\end{proof}
\end{thm}

This Theorem always
applies for some $i$, unless v is already sorted. Indeed,
it is sufficient to scan the array from left
to right, and return the first index i such
that v [ i ] > v [ i + 1 ] . This sweep is conducted by

Observe that Theorem~\ref{th_fat} always applies for some $i$, unless $v$ is already sorted.
Indeed, it is sufficient to scan the array from left to right,
and return the first index $i$ such that $v[i] > v[i+1]$.
This sweep is conducted by the function \textsc{UnfitOracle},
which either returns the index of an unfit element,
or determines that the array is already sorted,
in which case it returns~$-1$.
The pseudo-code of this function is given in Algorithm~\ref{alg:unfit_oracle}.

\begin{algorithm}
\caption{\textsc{Unfit Oracle}: Oracle able to find the index of an unfit array element}
\label{alg:unfit_oracle}      
\begin{algorithmic}[1]
\Function{\textsc{UnfitOracle}}{}
	\For {$i \leftarrow 0$ {\bf to} $n - 1$}
			\If{$v[i] > v[i+1]$}
				\Return {$ i $}
	\EndIf
	\EndFor
	\Return {$ -1 $}
\EndFunction
\end{algorithmic}
\end{algorithm}

Function \textsc{UnfitOracle}, combined with the \textsc{FixCycle} procedure, yields our sorting algorithm.
This algorithm runs \textsc{UnfitOracle} repeatedly, and,
as long as there is an unfit element in the array,
it fixes the position of that element along with its permutation cycle
by calling \textsc{FixCycle}.
This algorithm is realized by the procedure \textsc{OracleCyclicSort},
whose pseudo-code is given in Algorithm~\ref{alg:ocs_algorithm}.

\begin{algorithm}
\caption*{\textbf{Oracle Ciclic Sort}: Array sorting algorithm producing the minimum number of array writes}
%\label{alg1}      
\begin{algorithmic}[1]
\Procedure{OCS}{}
	\State {\bf static} $lp \leftarrow$ 0
	\State $next \textunderscore unfit \leftarrow UO(-1,lp)$
    \While {$next \textunderscore unfit \neq -1 $}
		\State $CPS(next \textunderscore unfit, lp)$
		\State $ fixed \leftarrow next \textunderscore unfit $
		\State $ next \textunderscore unfit \leftarrow UO(fixed,lp)$
	\EndWhile
\EndProcedure
\end{algorithmic}
\end{algorithm}


\section{Complexity and number of writes}

We now analize:
\begin{enumerate}[i.]
\item Total number of array writes.
\item Complexity of the algorithm.
\end{enumerate}

\subsection{Total number of array writes}

Only the \textsc{FixCycle} procedure writes elements into the array v.
Moreover, whenever it performs such a write, the element was unfit 
before the writing and becomes fit with the writting.
The total number of writes into the array $v$ is hence $|U(v)|$, which is 
clearly optimal.

\subsection{Complexity of the algorithm}

The \textsc{OracleCyclicSort} algorithm has a simple pseudocode made of a main loop that calls the \textsc{UnfitOracle} and \textsc{FixCycle}.

The \textsc{FixCycle} performs $n-1$ comparisons in order to find the correct position of an element $k$. To sort the array $v$, \textsc{FixCycle} have to find the correct position of every unfit element in the array. Therefore \textsc{FixCycle} have to resolve every array's permutation cycle.
Additional comparisons may be necessary just in case, a fit element $k$ would be passed to the procedure \textsc{FixCycle}. Nevertheless, according to the \textsc{UnfitOracle} property and the \textsc{OracleCyclicSort} algorithm, this case will never occur.
We can conclude that, if we call $u$ the total number of unfit elements, we have $u*(n-1)$ comparisons.

If we call $m$ the number of ciclic permutations in an array v of size n, we can say that $0 \le m \le u/2 \le n/2$, because one ciclic permutation have to be made at least of two unfit elements then there are at most $n/2$ differents permutation cycles in v.
We need less then $(n-1)$ comparison to find a permutation cycle, then a valid upperbound for \textsc{UnfitOracle} procedure is $(u/2)*(n-1) + (n-1)$ comparisons.

The \textsc{UnfitOracle} will ever perform globally less comparisons than \textsc{FixCycle}, so the final complexity is in order of $O(u*n)$.
In the best case we have an array already sorted, in fact the \textsc{OracleCiclicSort} main loop ends after the first \textsc{UnfitOracle} execution, so the lowerbound complexity is $\Omega(n)$.


% The conclusion

\section{General Case}

This is the conclusion

\begin{algorithm}[H]
\caption{\textsc{GeneralCorrectPositionOf}: Procedure that find the correct position of an element of v with repeated elements}
\label{alg:gen_corposof}      
\begin{algorithmic}[1]
\Procedure{\textsc{GeneralCorrectPositionOf}}{$i$, $val$}
	% Count the number $k$ of elements $ i < val$
	\State $k \gets 1$
	
	\For	{$j \gets 1$ {\bf to} $n$}
		\If{$j \neq i$ {\bf and} $v[j] < val$}
			\State $k \gets k + 1$
		\EndIf
	\EndFor
	
	\While{$val = v[k]$ \textbf{and} $k \neq i$}
		\State $k \gets k + 1$
	\EndWhile	
	
	\Return {$ k $}
\EndProcedure
\end{algorithmic}
\end{algorithm}

% The conclusion

\section{Conclusion}

This is the conclusion

%Text requiring further explanation\footnote{Example footnote}.

%----------------------------------------------------------------------------------------
%	REFERENCE LIST
%----------------------------------------------------------------------------------------

\printbibliography % aggiungo la bibliografia all'indice

%----------------------------------------------------------------------------------------

\end{document}
